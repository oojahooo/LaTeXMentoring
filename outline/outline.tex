% Created 2017-02-04 토 22:45
% Intended LaTeX compiler: pdflatex
\documentclass[11pt]{article}
\usepackage[utf8]{inputenc}
\usepackage[T1]{fontenc}
\usepackage{graphicx}
\usepackage{grffile}
\usepackage{longtable}
\usepackage{wrapfig}
\usepackage{rotating}
\usepackage[normalem]{ulem}
\usepackage{amsmath}
\usepackage{textcomp}
\usepackage{amssymb}
\usepackage{capt-of}
\usepackage{hyperref}
\usepackage{kotex}
\date{\today}
\title{}
\hypersetup{
 pdfauthor={},
 pdftitle={},
 pdfkeywords={},
 pdfsubject={},
 pdfcreator={Emacs 25.1.1 (Org mode 9.0.4)}, 
 pdflang={English}}
\begin{document}

\tableofcontents

\section{\TeX{} 멘토링 자료 outline}
\label{sec:orgb007ac3}
\subsection{intro}
\label{sec:org2de5bbd}
latex이 무엇인가?
왜 latex을 쓰는가?
기본적인 내용만 다루겠다.
\subsection{install}
\label{sec:org8195316}
latex은 어떻게 설치하고, 어떻게 시작하지?
초보자의 눈에서 천천히 하나씩
\subsection{start}
\label{sec:org9f622fe}
문서 작성을 시작하자
시작하려면 기본을 알아야지
\subsection{기능을 하나씩 알아보기}
\label{sec:org38cb18f}
여기서는 제작에 꼭 필요한 기능들만
주의할점? toc 여러번컴파일 필요할수도
environment가 꼬여서 연결되면안된다. 안에들어가는건 상관없는데 겹치는건안됨
명령어 구분방식?
명령어 사이에는 스페이스를 넣던지 해야됨 안그럼 하나로인식함

\subsection{기능을 더 구체적으로 알아보기}
\label{sec:orgfe24346}
documentclass에는 뭐가 있지? 각 documentclass의 역할은?(표를 뒤로 뺄까)
oblivoir 강조해주기
글씨크기, a4paper 옵션
\subsubsection{standalone, exam class 설명}
\label{sec:org48d3943}
\subsubsection{다양한 기본 기능 알아보기}
\label{sec:org7b4c559}
*글씨크기
*verbatim
*정렬(center, flushleft, flushright)
*는 굳이 설명안해도될듯
label \& reference
citation은 따로 설명하지 않겠다.
\begin{enumerate}
\item 수식넣기
\label{sec:org671eb64}
가장 중요
\end{enumerate}
\subsubsection{다양한 package 알아보기}
\label{sec:org21b3205}
geometry
indentfirst
setspace
amsmath
\subsubsection{표만들기 \& 그림넣기}
\label{sec:org6c6e510}
tabular \& graphicx
예시넣어서 설명해주기
\subsubsection{새로운명령어정의하기}
\label{sec:org4d53b5b}
\end{document}