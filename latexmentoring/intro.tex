%!TEX root = main.tex

\section*{들어가며}
이 자료는 한국과학영재학교 학생을 위한 \lt 멘토링 자료입니다.
입문자를 위해 설계되었으며, \lt 를 사용하는데 실제로 자주 사용되는 내용 위주로 구성하려고 노력했습니다.
내용의 흐름을 위해 자주 사용되지 않는 기능들에 대한 설명은 언급하지 않았습니다만, 그렇지 않더라도 \lt 의 수많은 기능들을 모두 언급할 수는 없는 점 이해해주셨으면 합니다.
이 글을 읽고 \lt 의 기본적인 사용 방법을 익히시고, 따로 필요한 기능들은 그때 그때 보다 전문적인 매뉴얼\footnote{\url{https://tobi.oetiker.ch/lshort/lshort.pdf}}을 참조하시거나, 인터넷에 검색하시며 사용하시면 됩니다.
이 글이 \lt 에 대한 진입장벽을 낮추는 데 도움이 되기를 기원합니다.
궁금한 점이 있으시면 ufolion@gmail.com 이나 oojahooo@gmail.com 로 연락해 주시면 답변 드리겠습니다.
\section{\lt 이해하기}
\label{sec:1}

\subsection{\lt 이란 무엇인가요?}
\label{sec:1.1}
\subsection*{}
\lt \footnote{라텍, 혹은 레이텍이라고 읽습니다}은 간단히 말해서 문서 작성 도구 중 하나입니다.
Microsoft Word나 한글과컴퓨터 한컴과 같이 우리가 컴퓨터상에서, 혹은 인쇄된 조판물로 보는 문서를 작성하기 위해 있는 시스템이라고 생각하시면 됩니다.

대신 앞서 언급한 워드, 한글과는 성격이 조금 다릅니다.
워드, 한글과 같은 프로그램은 WYSIWYG 편집기라고 부릅니다.
WYSIWYG란 What You See Is What You Get, 작성 중인 그대로 화면에 띄워서, 어떤 식으로 출력될지 보면서 편집하는 프로그램을 말합니다.
반면에 \lt 은 문서 작성과 결과 확인이 분리되어 있습니다.
파이썬 프로그래밍을 하듯이 문서에 대한 코드를 짜고, 이를 \lt 프로그램에 넣어서 실행시키면 우리가 볼 수 있는 PDF파일이 나온다고 생각하면 됩니다.
문서를 프로그래밍한다고 생각하셔도 되겠습니다.

문서를 어떻게 \emph{프로그래밍}하냐고요?
예를 들어 생각해봅시다.
우리가 워드프로세서에서 글을 크게 만들기 위해선 그 글을 드래그하고, 글씨를 키우는 버튼을 누르면 됩니다.
레이텍에서는 그 글 앞에 글씨를 크게하는 명령어를 붙이면 됩니다. \verb|\large Hello World!| 처럼요.
예상하셨겠지만 여기서는 \verb|\large|가 글씨를 크게 하는 명령어입니다.

\subsection{왜 \lt을 사용하나요?}
\label{sec:1.2}
\subsubsection*{}
그런데 \ref{sec:1.1}만 읽고 나면 왜 \lt 를 사용해야 하는지 의문이 들 것입니다.
워드프로세서로 보면서 만들면 되는데 왜 문서를 코딩까지 해야 하는지 말입니다.
\lt 이 워드프로세서가 개발되기 이전에 사용되던 구시대의 유물이라고 느껴질 수도 있겠죠.
하지만 \lt 은 지금도 이공계 다양한 분야에서 활발하게 사용되고 있는 도구입니다.
그렇다면 \lt 이  WYSIWYG 편집기에 비해 가지는 장점에는 무엇이 있길래 \lt 을 사용할까요?

\subsubsection{구조화\& 규격화된 문서}
\label{sec:1.2-org}
% bibtex, 글에 집중

우선 \lt 으로는 구조화되고, 규격화된 문서를 작성할 수 있습니다.
무슨 말이나고요?
수학 계열 학회 논문을 보시면 쉽게 이해하실 수 있으실 겁니다.
수식, 그림마다 붙어있는 번호, 복잡한 상호 참조, 1.1, 1.2.1, ... 논리적인 번호 매기기, 각주, 참고문헌까지.
같은 학회의 논문은 이 모든 구조가 똑같은 방식으로 이루어져 있습니다.
이 복잡한걸 어떻게 했냐고요?
사실 직접 한게 아니에요.
\lt 이 자동으로 해 준 거죠.

우선 줄간격, 여백 등의 기본적인 설정은 코드 몇 줄을 적어두는 걸로 충분합니다.
또한 \verb|section|, \\ \verb|subsection| 등의 명령어를 통해 문서를 정리해두기만 하면 번호는 \lt이 자동으로 매겨줍니다.
그림 역시 자동으로 번호가 매겨지고, 나중에 그 그림을 언급하고 싶을 때는 자동으로 그 그림에 맞는 번호를 찾아줍니다.
참고문헌은 bibtex와 같은 소프트웨어가 더욱 간단하게 규격에 맞게 reference를 작성할 수 있도록 도와주죠.

어쨌든 논문 하나를 적을 때마다 기본적인 설정을 코드로 설정해야 하니 워드나 한글에 비해 너무 귀찮지 않냐고요?
간단한 글을 작성할 때는 그런 걸 설정하지 않아도 충분하긴 하지만, 맞는 말이긴 합니다.
그렇지만 만약 그게 이미 다 작성되어 있다면 어떠시겠어요?
이미 모든 설정이 끝난 템플릿이 주어져 있다면, 거기에 쓰고 싶은 글만 채워 넣으면 되는 거 아니겠어요?
실제로 수학, 물리, 전산학 등의 많은 분야의 학회에서는  \verb|.tex|파일로 논문 양식을 제공합니다.
그럼 오히려 형식에 신경쓰지 않고, 쓰고자 하는 글에 집중할 수 있죠.
꼭 논문이 아니더라도 우리 학교 일물실 템플릿 하나만 만들어 놓으면, 혹은 친구에게 받으면, 그때부턴 \lt 으로 작성하는게 그다지 어렵지 않을 겁니다. 규격화 되고, 구조화 되어 있으니 결과물도 예쁘고요.

\subsubsection{간편한 수식 입력}
\label{sec:1.2-math}

\lt 은 수식 입력이 간편합니다.
수식 입력 상으로는 사실상 표준으로 자리잡았다고 할 수 있죠.
워드나 한글에서도 수식 입력이 된다고요?
사실 그것도 \lt 의 수식 입력 문법을 참조해서 만들어진 겁니다.
위키피디아에 있는 수식도 다 \lt 문법에 맞춰서 코딩된 겁니다.
그리고 워드나 한글에서 새로운 창을 띄워서 수식을 입력하고 닫는 복잡한 과정은 \lt 만큼 편하지도 않죠.
\lt 은 수식 입력이 간단하고, 만들어진 수식이 예쁩니다.
수식 입력에 관한 내용은 뒤에서 천천히 다루도록 하겠습니다.%상호참조

\subsubsection{안정성과 이식성}
\label{sec:1.2-steady}

\lt 파일은 \verb|.tex|라는 확장자로 저장되는데, 사실 이는 본질적으로 텍스트 파일입니다.
메모장으로도 수정할 수 있죠.
이 파일 안에는 저자가 문서를 어떻게 만들지에 대한 모든 내용이 알차게 들어 있습니다.
반면 워드나 한글 파일 같은 경우에는 저자가 적고자 하는 내용 이외에도 수많은 불필요한 정보들을 포함하고 있습니다. 그렇기 때문에 느리고, 무겁고, 깨지기도 하죠.
아마 대부분 한글 문서의 버전이 달라 파일이 깨진 것을 본 경험이 있을 겁니다.
그렇지만 텍스트 파일을 전달한다면, 혹은 \lt 으로 만들어진 pdf 파일을 전달한다면, 이러한 문제는 해결됩니다. 문서가 손상될 위험도 적고, 어느 컴퓨터에서나 열어볼 수 있죠.

마찬가지로 \lt 으로 문서를 작성하는 것 역시 편리한 점이 많습니다.
\lt 파일을 수정하는 것은 본질적으로 텍스트 파일을 수정하는 것과 같기 때문에 그다지 높은 컴퓨터 성능을 필요로 하지 않습니다.
500페이지 책을 쓴다고 생각해 보세요.
한글, 워드 프로그램으로 그걸 여는데 얼마나 오래 걸리겠습니까.
\lt 으로 만들면 마지막에 한번 pdf로 바꾸는 과정만 거치면 됩니다.

\subsubsection{수많은 확장 기능}
\label{sec:1.2-ext}

여러분은 왜 Python이 많이 사용되는지 아시나요?
Python은 쉽기도 하지만, 다양한 확장 기능을 통해 미분방정식, 데이터 처리 등 수많은 일을 할 수 있다는 장점이 있습니다.
한번쯤 경험해 보셨을 수 있겠지만, \verb|import|를 통해 라이브러리를 불러와서 이런 일들을 하곤 합니다.
\lt 역시 비슷합니다.
뒤에서 설명드리겠지만 \lt 에서는 \verb|\usepackage{}|가 비슷한 역할을 합니다.
\lt 의 다양한 패키지를 활용하면 워드프로세서의 다양한 편의 기능을 구현하는 것은 물론, 프레젠테이션 자료, 악보, 그래프 그리기 등의 수많은 기능을 구현할 수 있습니다. 

아 그리고 아까 \lt 이 문서를 \emph{프로그래밍}한다고 묘사했죠?
\lt 에서는 문서를 작성할 때 조건문을 사용하거나, 새로운 명령(\verb|\newcommand|)을 만들어 이용할 수 있기 때문에, 배우시다 보면 상당히 편하게 이용하실 수 있으실 겁니다. 

\subsubsection*{}
아 그리고 한 가지가 더 있네요.
\lt 은 무료입니다.
누구나 다운로드받아서 사용할 수 있죠.
이 외에도 미학적\footnote{\url{http://nitens.org/taraborelli/latex}}, 철학적\footnote{\url{http://ricardo.ecn.wfu.edu/~cottrell/wp.html}}인 관점에서 \lt 의 다양한 장점을 찾으실 수 있습니다.
처음에만 어렵게 느껴질 뿐, 익숙해지면 여러분도 왜 \lt 을 사용하는지 금방 알게 되실 겁니다.

%%% Local Variables:
%%% mode: latex
%%% TeX-master: "main"
%%% End:
