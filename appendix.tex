%!TEX root = main.tex

\newpage
\section{Appendix}
\label{appendix}
\subsection{documentclass}
\label{subsec:dctable}
\begin{table}[h]
	\centering
	\begin{tabular}{|c|m{250pt}|}
		\hline
		article & 과학 저널, 프레젠테이션, 짧은 보고서, 프로그램 문서, 초대장 등의 글을 쓸 때 사용합니다. \\
		\hline
		IEEEtran & IEEE\footnote{Institute of Electrical and Electronics Engineers, 미국 전기 전자 학회} 양식에 맞도록 글을 쓸 때 사용합니다.\\
		\hline
		proc & article 중 회의록을 작성하기 위해 사용합니다.\\
		\hline
		report & Chapter를 포합하는 긴 보고서, 작은 책, 학위 논문 등에 쓰입니다.\\
		\hline
		book & 책자를 만들 때 사용합니다.\\
		\hline
		slides & 슬라이드를 만들 때 사용합니다. big sans serif 폰트를 사용합니다.\\
		\hline
		memoir & book class보다 개선된 책자 레이아웃입니다.\\
		\hline
		letter & letter를 쓸 때 사용합니다.\\
		\hline
		beamer & 발표자료를 만들 때 사용합니다.\\
		\hline
	\end{tabular}
        \caption{Document Classes}
        \label{tab:doc cls}
\end{table}

\begin{table}[h]
	\centering
	\begin{tabular}{|c|m{320pt}|}
		\hline
		10pt, 11pt, 12pt & 문서의 글자 크기를 결정합니다. 아무 옵션도 적용하지 않으면 10pt로 간주합니다.\\
		\hline
		a4paper, letterpaper,\dots & 용지 크기 및 모양을 설정합니다. 기본은 letterpaper입니다. 다만 상황에 따라 기본 크기가 다르게 설정될 수 있으므로, 될 수 있으면 직접 명시하는 것이 좋습니다.\\
		\hline
		fleqn & 기본 가운데 정렬로 되어 있는 수식을 왼쪽정렬합니다. 후에 자세히 설명해드리겠지만, 수식만 따로 쓰는 경우 기본 가운데 정렬입니다. 이 때 이 option이 있으면 왼쪽으로 정렬됩니다.\\
		\hline
		leqno & 역시 후에 자세히 다루겠지만, 수식을 쓸 때 줄마다 numbering이 됩니다. 이 때 기본 오른쪽에 numberimg이 뜨는 것을 왼쪽으로 바꿔줍니다.\\
		\hline
		titlepage, notitlepage & 표지를 따로 만들지 말지를 결정합니다. article은 없는 것이, report나 book은 있는 것이 기본입니다.\\
		\hline
		twocolumn & 2단 편집을 할 수 있도록 합니다. 왼쪽 오른쪽 단을 나눠서 쓰는 것을 생각하면 됩니다.\\
		\hline
		twoside, oneside & 양면인쇄할지 단면인쇄할지 결정합니다. article이나 report에서는 단면인쇄, book에서는 양면인쇄를 기본으로 합니다. 이 설정은 프린트 설정에 따라 적용이 안 되거나 문제가 생길 수 있으니 주의해주세요.\\
		\hline
		landscape & 용지 방향을 가로로 바꿔줍니다.\\
		\hline
		openright, openany & 새로운 장을 항상 오른쪽 페이지에서 시작할지, 바로 다음 새 페이지에서 시작할지 설정합니다. article에서는 chapter가 존재하지 않아 쓸 수 없습니다. report에서는 새 페이지, book에서는 오른쪽 페이지가 기본입니다.\\
		\hline
	\end{tabular}
    \caption{Document Class Options}
    \label{tab:doc cls opt}
\end{table}

\newpage
\subsection{수식 모드 공백 기호}
\begin{table}[hp]
	\centering
	\begin{tabular}{|c|c|}
		\hline
		\verb|\,| & 전각의 $\frac{3}{18}$\\
		\hline
		\verb|\:| & 전각의 $\frac{4}{18}$\\
		\hline
		\verb|\;| & 전각의 $\frac{5}{18}$\\
		\hline
		\verb|\ | & 반각\\
		\hline
		\verb|\quad| & 전각\\
		\hline
		\verb|\qquad| & 전각의 2배\\
		\hline
	\end{tabular}
	\caption{강제 공백 기호들}
	\label{tab:quad}
\end{table}

\subsection{글 위, 아래 기호}
\begin{table}[hp]
	\centering
	\begin{tabular}{|c|c|}
		\hline
		\verb|\overrightarrow{text}|, \verb|\overleftarrow{text}| & text 위에 오른쪽, 왼쪽 화살표를 그립니다.\\
		\hline
		\verb|\overleftrightarrow{text}| & text 위에 양쪽 화살표를 그립니다.\\
		\hline
		\verb|\overline{text}|, \verb|\underline{text}| & text 위, 아래에 수평선을 그립니다.\\
		\hline
		\verb|\overbrace{text}|, \verb|\underbrace{text}| & text 위, 아래에 중괄호를 그립니다.\\
		\hline
	\end{tabular}
	\caption{글자 위, 아래에 그려지는 기호들}
	\label{tab:overunder}
\end{table}

\subsection{표 열의 내용 정렬하기}
\begin{table}[hp]
	\centering
	\begin{tabular}{|c|c|}
		\hline
		l & 왼쪽 정렬된 열\\
		\hline
		c & 가운데 정렬된 열\\
		\hline
		r & 오른쪽 정렬된 열\\
		\hline
		\verb|p{`width'}| & `width'에 맞춰 양쪽정렬 및 자동줄바꿈을 하며 위쪽정렬합니다.\\
		\hline
		\verb|m{`width'}| & `width'에 맞춰 양쪽정렬 및 자동줄바꿈을 하며 가운데정렬합니다.(array package 필요)\\
		\hline
		\verb|b{`width'}| & `width'에 맞춰 양쪽정렬 및 자동줄바꿈을 하며 아래쪽정렬합니다.(array package 필요)\\
		\hline
		$\mid$ & 세로줄\\
		\hline
		$\mid$$\mid$ & 세로 두 줄\\
		\hline
		\multicolumn{2}{c}{*`width'는 열의 너비로 cm, pt 혹은 textwidth command를 사용하여 나타냅니다.}
	\end{tabular}
	\caption{정렬하는 명령어들}
	\label{tab:pos}
\end{table}

%%% Local Variables:
%%% mode: latex
%%% TeX-master: "main"
%%% End:
