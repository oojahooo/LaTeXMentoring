\documentclass{article}
\usepackage{kotex}
\usepackage{setspace}
\usepackage{amsmath}
\usepackage{amssymb}
\usepackage[left=2.5cm,right=2.5cm,top=3cm,bottom=3cm,a4paper]{geometry}
\begin{document}
	\doublespacing
	\noindent 수식을 입력해 봅시다.\\
	막 내용을 입력하다 중간에 짧은 수식을 넣을 수 있습니다.\\
	예를 들어 $c \in A$ 이렇게 말입니다.\\
	조금 길거나 큰 수식을 따로 입력할 수도 있습니다.\\
	예를 들어
	\[\left\{ \begin{array}{ll}
		a \in f(a)
		\\a \notin f(a)
	\end{array} \right.\]
	이렇게 말입니다.\\
	심지어
	\begin{align}
		|f(x)|=|sf_1(x)+tf_2(x)| &\le |sf_1(x)|+|tf_2(x)| \nonumber
		\\&=|s||f_1(x)|+|t||f_2(x)| \nonumber
		\\&\le |s|c_1|g(x)|+|t|c_2|g(x)| \nonumber
		\\&=(|s|c_1+|t|c_2)|g(x)| \nonumber
	\end{align}
	이런 것도 됩니다.\\
	\LaTeX 은 정말 많은 수식 요소들을 지원해줍니다.\\
	$a^{bcd}_{efg}$와 같은 첨자는 물론이고, $\sum$, $\int$같은 것도 지원합니다.
	\[\frac{\pi}{2}\]
	이렇게 분수도 잘 됩니다.
\end{document}